% ##################################################
% Dokumentvariablen
% ##################################################

% Persoenliche Daten
\newcommand{\docNachname}{Nachname}
\newcommand{\docVorname}{Vorname}
\newcommand{\docStrasse}{Straße}
\newcommand{\docOrt}{Ort}
\newcommand{\docPlz}{PLZ}
\newcommand{\docEmail}{Mail}
\newcommand{\docMatrikelnummer}{Matrikel}

% Dokumentdaten
\newcommand{\docTitel}{Titel}
\newcommand{\docUntertitel}{Untertitel}
\newcommand{\docArtDerArbeit}{Art der Arbeit}
\newcommand{\docStudiengang}{Studiengang}
\newcommand{\docAbgabedatum}{Abgabedatum}
\newcommand{\docErsterReferent}{Erster Referent}
\newcommand{\docZweiterReferent}{Zweiter Referent}

% ##################################################
% Allgemeine Pakete
% ##################################################

% Abbildungen einbinden
\usepackage{graphicx}
\usepackage{minipage-marginpar}

% Paket zur Index-Erstellung (Schlagwortverzeichnis)
\usepackage{imakeidx}
\makeindex[intoc, title=Schlagwortverzeichnis, columns=1]
\indexsetup{
	headers={\indexname}{\indexname}
}

% Farben
\usepackage{color}
\usepackage[usenames,dvipsnames,svgnames,table]{xcolor}

% Maskierung von URLs und Dateipfaden
\usepackage[hyphens]{url}

% Deutsche Unterstützung
\usepackage[ngerman]{babel}

% Mathematische Zeichen etc.
\usepackage{amsmath}
\usepackage{amssymb}

% ##################################################
% Seitenformatierung
% ##################################################

\KOMAoption{paper}{A4}
\KOMAoption{twoside}{true}

\usepackage[
	portrait,
	top = 3cm,
	bottom = 2cm,
	inner = 2.5cm,
	outer = 2.5cm,
	bindingoffset = 1.5cm,
]{geometry}


% ##################################################
% Kopf- und Fusszeile
% ##################################################

\usepackage[singlespacing = true]{scrlayer-scrpage}
\clearpairofpagestyles

\KOMAoption{headsepline}{true}
\KOMAoption{plainheadsepline}{true}

\ohead*{\pagemark} % Seitennummer außen
\ihead*{\leftmark} % Kapitel innen

\setkomafont{pageheadfoot}{\bfseries} % Kopfzeile fett
\setkomafont{pagination}{\bfseries} % Seitenzahl fett


% ##################################################
% Absätze
% ##################################################

\KOMAoption{parskip}{half}

% Absätze der Überschriften anpassen
\RedeclareSectionCommand[afterskip = \baselineskip]{chapter}
\RedeclareSectionCommand[beforeskip = \baselineskip, afterskip = 1sp]{section}
\RedeclareSectionCommand[beforeskip = \baselineskip, afterskip = 1sp]{subsection}

% ##################################################
% Schriften
% ##################################################

% Erlaubt beliebig große Schrift
\usepackage{lmodern}

% Schriftgröße
\KOMAoption{fontsize}{12pt}

% Stdandardschrift festlegen
\renewcommand*{\familydefault}{\sfdefault}

% Standard Zeilenabstand: 1,5 zeilig
\usepackage{setspace}
\onehalfspacing 

% Schriftgroessen festlegen
\setkomafont{chapter}{\large\bfseries} 
\setkomafont{section}{\normalsize\bfseries} 
\setkomafont{subsection}{\normalsize\mdseries} 
\setkomafont{caption}{\normalsize\mdseries} 

% ##################################################
% Inhaltsverzeichnis / Allgemeine Verzeichniseinstellungen
% ##################################################

\KOMAoption{toc}{chapterentrywithdots}
\KOMAoption{listof}{nochaptergap, entryprefix}

% Einträge für andere Verzeichnisse
\setuptoc{toc}{totoc}
\KOMAoption{toc}{listof, index, bibliography}

% Schriftart und -groesse im Inhaltsverzeichnis anpassen
\setkomafont{chapterentry}{\normalsize}

% Zeilenabstand in den Verzeichnissen einstellen
\BeforeStartingTOC[toc]{
	\DeclareTOCStyleEntry[
		beforeskip=.5\baselineskip
	]{tocline}{chapter}
	
	\DeclareTOCStyleEntry[
		beforeskip=.5\baselineskip
	]{tocline}{section}
	
	\renewcommand*{\autodot}{}
}


% ##################################################
% Allgemeine Abbildungs- und Tabellenverzeichnis Optionen
% ##################################################

% Pakete für Tabellen und Abbildungen
\usepackage{chngcntr}
\usepackage{caption}
\usepackage{subcaption}

% ##################################################
% Abbildungsverzeichnis und Abbildungen
% ##################################################

\usepackage{wrapfig}

% Nummerierung von Abbildungen
\counterwithout{figure}{chapter}

\BeforeStartingTOC[lof]{
	\DeclareTOCStyleEntry[
		beforeskip=.5\baselineskip,
		numwidth=10em
	]{tocline}{figure}
	
	\renewcommand*{\autodot}{:}
}


% ##################################################
% Tabellenverzeichnis und Tabellen
% ##################################################

\usepackage{booktabs}

% Nummerierung von Tabellen
\counterwithout{table}{chapter}

\BeforeStartingTOC[lot]{
	\DeclareTOCStyleEntry[
		beforeskip=.5\baselineskip,
		numwidth=10em
	]{tocline}{table}
	
	\renewcommand*{\autodot}{:}
}


% ##################################################
% Listings (Quellcode)
% ##################################################

\usepackage{minted}
\usepackage[babel, german=quotes]{csquotes}
\MakeOuterQuote{"}

\renewcommand{\listoflistingscaption}{Quellcodeverzeichnis}
\renewcommand{\listingscaption}{Quellcode}
\newcommand{\listingautorefname}{Quellcode}

\renewcommand*{\listoflistings}{\listoftoc[{\listoflistingscaption}]{lol}}
\setuptoc{lol}{totoc}

\BeforeStartingTOC[lol]{
	\DeclareTOCStyleEntry[
		level=1,
		indent=1.5em,
		numwidth=10em,
		beforeskip=.5\baselineskip,
		entrynumberformat=\listingscaption~,
	]{tocline}{listing}
	
	\renewcommand*{\autodot}{:}
}

% ##################################################
% Theoreme
% ##################################################

% Umgebung für Formeln
\newtheorem{formel}{Formel}
\newcommand{\formelautorefname}{Formel}
  	
% ##################################################
% Literaturverzeichnis
% ##################################################

\usepackage[
	backend = biber,
	bibstyle = authoryear,
	citestyle = authoryear,
	giveninits = true,
	uniquename = false,
	dashed = false,
]{biblatex}

\DeclareNameAlias{sortname}{family-given}

\DefineBibliographyStrings{ngerman}{
	bibliography = {{Literaturverzeichnis}},
	andothers = {{et al.}},
	mathesis = {{Masterarbeit}},
}

\addbibresource{meta/literature.bib}

% ##################################################
% PDF / Dokumenteninterne Links
% ##################################################

\usepackage[
	colorlinks=false,
   	linkcolor=black,
   	citecolor=black,
  	filecolor=black,
	urlcolor=black,
    bookmarks=true,
    bookmarksopen=true,
    bookmarksopenlevel=3,
    bookmarksnumbered,
    plainpages=false,
    pdfpagelabels=true,
    hyperfootnotes,
    pdftitle ={\docTitel},
    pdfauthor={\docVorname \docNachname},
    pdfcreator={\docVorname \docNachname}
]{hyperref}

% ##################################################
% Abkürzungen
% ##################################################

\usepackage[acronym, nogroupskip, nonumberlist, nopostdot, translate=babel]{glossaries}
\makenoidxglossaries
\loadglsentries{meta/glossary}

% Fettgedruckte Abkürzung im Verzeichnis
\renewcommand*{\glsnamefont}[1]{\textbf{#1}}


% ##################################################
% Anhang
% ##################################################

\newcommand*{\appendixstyle}{
	\renewcommand*{\raggedchapter}{\centering}
	\RedeclareSectionCommand[font=\Huge, beforeskip=0.4\textheight]{chapter}
	
	\RedeclareSectionCommand[font=\large, beforeskip=0pt, afterskip=1sp]{section}
	
	\RedeclareSectionCommand[font=\bfseries, beforeskip=0pt, afterskip=1sp]{subsection}
}
